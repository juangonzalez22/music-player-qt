\documentclass[a4paper,12pt]{article}

\usepackage[utf8]{inputenc}
\usepackage{amsmath}
\usepackage{hyperref}
\usepackage{url}
\usepackage{breakurl}

\title{Manual de Usuario - JL MediaPlayer}
\author{}
\date{}

\begin{document}

\maketitle

\section*{Introducción}

JL MediaPlayer es un reproductor multimedia que ofrece una experiencia sencilla y fluida para reproducir tus archivos de audio y video. Desarrollado por \textbf{juangonzalez22} y \textbf{LoidherJaimes} de la Universidad Tecnológica de Pereira, este programa permite una amplia gama de reproducción de formatos, tanto de audio como de video, todo con una interfaz fácil de usar.

Este manual te guiará en la instalación, el uso y la solución de problemas de JL MediaPlayer.

\section*{Requisitos del Sistema}

\begin{itemize}
    \item \textbf{Sistema operativo:} Windows
    \item \textbf{Tamaño del archivo de instalación:} 30 MB
    \item \textbf{Espacio en disco:} Recomendado al menos 90 MB libres
    \item \textbf{Requisitos adicionales:} Ninguno
\end{itemize}

\section*{Instalación}

Para instalar JL MediaPlayer en tu computadora, sigue estos pasos:

\begin{enumerate}
    \item \textbf{Descargar el instalador} desde el repositorio oficial: \url{https://github.com/juangonzalez22/music-player-qt}
    \item Ejecuta el archivo \texttt{JLMediaPlayer.exe}.
    \item Selecciona el idioma para la instalación.
    \item Acepta el acuerdo de licencia y haz clic en "Siguiente".
    \item Elige la carpeta donde se instalará el programa.
    \item Si deseas, selecciona crear un acceso directo en el escritorio.
    \item Haz clic en "Instalar" y espera a que el proceso se complete.
    \item Al finalizar, haz clic en "Finalizar" para completar la instalación.
\end{enumerate}

\textbf{Nota importante}: Durante la instalación, algunos programas de seguridad como antivirus o firewalls podrían bloquearla. Si esto sucede, desactívalos temporalmente y vuelve a intentarlo.

\section*{Desinstalación}

Para desinstalar JL MediaPlayer, sigue estos pasos:

\begin{enumerate}
    \item Abre el \textbf{Panel de Control} de Windows.
    \item Selecciona \textbf{Desinstalar un programa}.
    \item Busca \textbf{JL MediaPlayer} en la lista y haz clic en él.
    \item Confirma que deseas desinstalar el programa.
    \item El proceso se completará automáticamente y recibirás una confirmación.
\end{enumerate}

\section*{Interfaz de Usuario}

La interfaz de JL MediaPlayer se divide en tres secciones principales para facilitar su uso:

\subsection*{Explorador de Archivos}

Aquí es donde puedes seleccionar y cargar los archivos para reproducir:

\begin{itemize}
    \item \textbf{Botón "Abrir"}: Abre el explorador de archivos de Windows para seleccionar un archivo.
    \item \textbf{Explorador de carpetas}: Navega por las carpetas de tu sistema.
    \item \textbf{Lista de archivos}: Muestra los archivos de la carpeta seleccionada que pueden ser reproducidos.
\end{itemize}

\subsection*{Reproductor de Video}

Esta sección permite visualizar y escuchar el contenido multimedia:

\begin{itemize}
    \item \textbf{Reproducción de audio}: Muestra una representación de las ondas sonoras mientras se reproduce el audio.
    \item \textbf{Reproducción de video}: Muestra el video en su formato original.
\end{itemize}

\subsection*{Panel de Control}

El Panel de Control es donde encontrarás los botones para gestionar la reproducción de tus archivos:

\begin{itemize}
    \item \textbf{Botón de Reproducir/Pausar}: Inicia o pausa la reproducción del archivo seleccionado.
    \item \textbf{Botón de Detener}: Detiene la reproducción y reinicia el archivo desde el principio.
    \item \textbf{Botones de Anterior y Siguiente}: Permiten moverse entre archivos en la lista de reproducción.
    \item \textbf{Botón de Volumen/Silencio}: Ajusta el volumen o silencia el sonido.
    \item \textbf{Slider de Volumen}: Permite ajustar el nivel de volumen.
    \item \textbf{Slider de Tiempo}: Muestra el progreso de la reproducción y te permite saltar a un momento específico en el archivo.
    \item \textbf{Botón de bucle}: Alterna entre los modos de bucle (repite toda la playlist), bucle único (repite únicamente el archivo escogido) y bucle desactivado.
    \item \textbf{Botón de modo aleatorio}: Activa o desactiva el modo aleatorio.
\end{itemize}

\section*{Funciones Adicionales}

\begin{itemize}
    \item \textbf{Carpetas como listas de reproducción}: Las carpetas en tu sistema funcionan como listas de reproducción, facilitando la organización y selección de archivos.
    \item \textbf{Reproducción en segundo plano}: JL MediaPlayer puede seguir reproduciendo tus archivos mientras está minimizado o en segundo plano.
    \item \textbf{Sin atajos de teclado}: No se han configurado atajos de teclado en JL MediaPlayer.
    \item \textbf{Compatibilidad con formatos}: Actualmente, JL MediaPlayer soporta los siguientes formatos:
    \begin{itemize}
        \item \textbf{Audio:} MP3, WAV, M4A
        \item \textbf{Video:} MP4, M4V, MKV
    \end{itemize}
\end{itemize}

\section*{Solución de Problemas}

\begin{enumerate}
    \item \textbf{Si un archivo no se reproduce:} Verifica que el archivo esté en un formato compatible y no esté dañado. Si el programa no responde, cierra la aplicación y vuelve a abrirla. No perderás datos ni configuraciones.
    \item \textbf{Si no hay sonido:} Asegúrate de que el volumen del sistema esté correctamente configurado. Si tienes programas de seguridad como antivirus o firewalls bloqueando la reproducción, desactívalos temporalmente.
    \item \textbf{Problemas con la instalación:} Si la instalación no se completa, revisa si hay antivirus o firewalls bloqueando el proceso e intenta desactivarlos antes de reiniciar la instalación.
\end{enumerate}

\section*{Información Adicional}

\begin{itemize}
    \item \textbf{Actualizaciones:} JL MediaPlayer no se actualiza automáticamente. Para comprobar si hay actualizaciones disponibles, visita el repositorio en \url{https://github.com/juangonzalez22/music-player-qt}.
    \item \textbf{Personalización y formatos adicionales:} En este momento, no es posible agregar más formatos ni personalizar la interfaz de usuario.
    \item \textbf{Distribución y modificación:} Los usuarios pueden distribuir, modificar y mover JL MediaPlayer libremente, siempre que se dé crédito a los autores originales.
\end{itemize}

\section*{Agradecimientos}

Agradecemos a todos los usuarios que han probado y utilizado JL MediaPlayer, así como a aquellos que nos han brindado su apoyo en el desarrollo de la aplicación.

\section*{Licencia}

JL MediaPlayer es un software gratuito (\textbf{freeware}). Puedes usarlo, distribuirlo y modificarlo según lo desees, siempre que se dé crédito a los autores originales.

\section*{Soporte Técnico}

Si necesitas ayuda o tienes preguntas, puedes ponerte en contacto con los desarrolladores a través de sus perfiles de GitHub:

\begin{itemize}
    \item \url{https://github.com/juangonzalez22}
    \item \url{https://github.com/LoidherJaimes}
\end{itemize}

\end{document}
